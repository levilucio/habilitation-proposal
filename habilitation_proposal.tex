% %This is a very basic article template.
% %There is just one section and two subsections.
\documentclass{scrartcl}

\usepackage[ colorlinks=true, urlcolor=blue, linkcolor=green ]{hyperref}
\usepackage[title]{appendix}
\usepackage{graphicx}

\begin{document}

\title{Evaluation of Domain-Specific Languages in Practice\\}
\subtitle{Proposal for a Habilitation at the Technische Universit\"at M\"unchen}

\author{Dr. Levi L\'ucio}

\maketitle

\abstract{In this document we introduce our research topic of interest for
pursuing a habilitation at the Technische Universit\"at M\"unchen.}

\section{Document Structure}

We will start by providing background information on the topic of
Domain-Specific Languages in section~\ref{sec:background}. We then go on  in
section~\ref{sec:soa} to discuss the state of the art of the domain and to
highlight the running trends during the last two decades. In
section~\ref{sec:topic} we introduce our topic of interest and finally in
section~\ref{sec:context4dev} we lay out the operational context under which our
projected research will take place.

\section{Background}
\label{sec:background}

Computer languages are very often domain-specific, in the sense that they
specialize in enabling the description of specifications, operational or not, of
certain technical artifacts. Domain-Specific Languages (DSLs) are often
presented in opposition to General-Purpose Languages (GPLs), which are
all-purpose generic programming languages such as Java or C. The main premise
for DSL construction is that such languages should explicitly address a domain of
interest by taking the concepts of that domain first-class
citizens~\cite{Kolovos06}. This can imply a trade-off in generality, meaning
some DSLs may not be Turing-complete as such computational power may not be
required for a specific application. The advantages put forward by the
proponents of the DLS approach are increased productivity for language users and reduced
maintenance costs of DSL programs. Additionally, DSLs are expected to lower the
barrier of computer language usage for non-experts, thus allowing users from
other domains to more easily specify the artifacts and computations they require~\cite{MernikHS05}.

DSLs have a rich history of contributions in different areas of knowledge.
Some notable examples of DSLs that left permanent marks in their domains are
\emph{lex} and \emph{yacc}~\cite{Johnson80} for compiler construction,
HTML~\cite{BernersLee96} as a markup language for the world-wide web,
VHDL~\cite{Ashenden02} for hardware specification, Excel for spreadsheets, Latex~\cite{Lamport1989} for
typesetting, SQL~\cite{Codd70} for database querying or MATLAB~\cite{Gilat07}
for technical computing.

During the past decade a number of DSL workbenches saw the light of day. The
Eclipse Modelling Framework (EMF)~\cite{emf} had significant impact in the
academia and became one of the most popular frameworks for DSL construction.
More recently, the MPS workbench from Jetbrains~\cite{mps} has delivered a
powerful DSL construction workbench with professional support and possibilities
to design attractive GUIs that provide a very interesting complement to the
notion of domain specificity. Professional DSL workbenches such as
MetaEdit+~\cite{metaedit} have repeatedly made strong cases for the industrial
adoption of DSLs and have carved a niche market in the domain.

During the past few years, DSLs have raised considerable interest among software
developers~\cite{Fowler10}. The paradigm speaks directly to engineers who wish
to build their systems ground-up while making as much use as possible of domain
knowledge that is typically hard-earned. In this context, software clients often
regard DSLs as key-in-hand solutions. They encompass simplified means to express
domain-specific computations while abstracting from accidental complexity linked
to the software or the hardware running underneath.

Companies such as itemis AG, PROTOS GmbH or MetaCase have successfully developed
business models around DSLs. They leverage their knowledge of modelling and DSL
workbenches to deliver to customers key-in-hand software solutions. Such
solutions have been developed for disparate domains such as the automotive,
avionics, power tools, health, biology, among many others. In practice they help
either software developers or final users in achieving their tasks.

Despite these successes, the potential and real impact of DSLs in industry is
still ill understood. A recent compelling report from Tolvanen and
Kelly~\cite{Tolvanen016} states that their company, MetaCase, can affirm with
certainty that the gains in productivity of their clients range from 5-fold to
10-fold. In the same article, the authors go on to claim that academic research
in the domain has thus far not been unable to validate such gains in general.

Anecdotally, at the PAINS workshop at MODELS 2018~\cite{pains18}, an interesting
discussion raged between a high-profile DSL proponent and a top-level BMW manager. While
the DSL proponent insisted that the (MPS-based) technology was ready and could
serve as a ``silver bullet'' of sorts, the BMW manager replied that the attempts of using
DSLs at his company where ``hit-and-miss'', and that even when DSLs did prove
successful, it was not understood why. A question from that same manager that
was particularly thought-provoking was: ``how do I build an abstraction and
know that it is a good one?''. A more general criticism to the DSL approach in
general was: ``quality standards are missing, how do I to judge or trust your
processes of building DSLs while making decisions that will benefit my company?''.
 
\section{State of the Art}
\label{sec:soa}

In 2000, a landmark white paper from the company MetaCase
claimed that, using their commercial tool MetaEdit+, the company developed DSLs
for Nokia increased their software productivity 10-fold~\cite{metacase00}.
Such early claims spawned a large amount of interest in DSLs by both the
industry and academia.

Following such interest, authors such as Spinellis or V\"olter pursued the idea
of proposing patterns for DSLs construction~\cite{Spinellis01,VolterB04}, much
in the light of what had been very successfully done by the ``gang of four'' for
design patterns~\cite{Gamma:95}. Such patterns concentrate mainly best practices
for DSL design and implementation. This research culminated in the very
well-cited work of Mernik \emph{et al.}~\cite{MernikHS05}, which builds on the
work of Spinellis and aims at enabling the DSL developer with the understanding of
where and when design and implementation decisions matter in DSL development.
Also, reports on industrial best practices in DSL usage such as the work from
Wile~\cite{Wile03} started to emerge at this time.

The raise in interest in DSLs also brought upon a number of DSL workbenches,
both open-source and commercial, such as the Eclipse Modelling Framework
(EMF) for Eclipse~\cite{emf}, Microsoft's DSL tools~\cite{microsoftDSLTools},
GME~\cite{gme} or AToM3~\cite{atom3}. In particular, EMF was heavily adopted by
academia for pursuing research on DSLs and together with GMF~\cite{gmf} soon
became the platform of choice for academic DSL development.
 
While the enthusiasm around DSLs continued strong throughout the 00s and work on
how to develop good languages kept on being published~\cite{Voelter09}, other
authors began to question the state of practice in general. In particular Kelly
from MetaCase raises in~\cite{Kelly2009} the point that DSL development often
takes a standpoint of self sufficiency, ignoring pitfalls that mostly have to do
with taking the domain for which the language is developed or its final users
too lightly. This was in the same year corroborated by work from Gabriel \emph{et
al.}~\cite{Gabriel09} who reviewed a significant amount of literature related to
the evaluation of DSLs. The authors claim that, while the benefits of DSLs in
terms of increased productivity in well-defined domains were often put forward
in the literature, little to no evidence to back up such claims was offered.

Having recognized the fact that DSLs were critically missing supporting
evaluation that would back up the promises from the early days, authors such as
Kolovos and Paige~\cite{Kolovos06} proposed sets of characteristics that
``good'' DSLs should exhibit. Examples of such generic characteristics are
conformity to the domain of choice, supportability by tools, integrability with
other languages, longevity, simplicity, quality, scalability, learnability or
usability.
Several studies from the late 00's and beginning of the 10's have concentrated
on proposing means to quantitatively assess such and other
characteristics~\cite{KellyTolvanen09,Hermans09,Barisic:12,Kahraman2015}. A
notable study dating from this period from Kosar \emph{et al.}~\cite{Kosar2012}
provides empirical evidence that programs written using DSLs are more easily
comprehensible by programmers than programs written using GPLs, one of the
original claims of the DSL community.

During this same period several contemporary DSL workbenches made their first
appearance: the Meta Programming System (MPS)\footnote{Note that while the MPS
project exists since 2003, its mainstream appearance dates to a decade later.}
from JetBrains~\cite{mps}, Sirius (used by Obeo as a core
technology~\cite{sirius}), Xtext from itemis AG~\cite{xtext}, among
others~\cite{Kelly:2013}. Benefiting from the previous decade of development of
the DSL domain and professional software development teams, these tools exhibit
a level of maturity that allows them to be used in industrial settings with
success. For instance Sirius, based on the Eclipse platform, has been used to
develop DSLs for the aerospace, transportation or energy industries~\cite{obeo}.
MPS~\cite{mps}, one of the first DSL workbenches using projectional editing, has
been used to develop commercial tools for branches such as software development,
health, insurance companies and automotive, among others.
Xtext~\cite{xtext}, a language for developing textual DSLs, is currently used by
large players such as Google, SAP or Bosch.

While the above seems to indicate that DSLs have successfully permeated the
industry, authors from recent surveys remain very critical of the actual state
of the art. Mernik~\cite{Mernik17} published a systematic mapping study in
2017 where he shows that the \emph{maintenance} and \emph{validation} phases of
DSL development are grossly under-studied in the literature when compared to
\emph{domain analysis}, \emph{design} and \emph{implementation} (see
figure~\ref{fig:DSL_design_phases}).

\begin{figure}[!h]
\centering 
\includegraphics[width=1\textwidth]{./figures/DSL_design_phases}
\caption{Distribution of 810 scientific contributions according to the phases
of DSL construction they cover (extracted from~\cite{Mernik17})}
\label{fig:DSL_design_phases}
\end{figure}

Equally, Tolvanen and Kelly who have been active members DSL community since its
inception, have recently published a summary of their experiences with their
commercial tool MetaEdit+~\cite{TolvanenKelly2016}. There, they state that
although they can consistently claim a 5 to 10 times improvement in productivity for their customers, the DSL community has not been able to do the same in
general. Note that this does not contradict the paragraph above on adoption by
the industry, as: 1) adoption does not mean increased productivity, and 2) it is
to be expected that if productivity increases do exist, corporate secrecy would
be put in place to avoid losing competitive edges. Tolvanen and Kelly do
mention in~\cite{TolvanenKelly2016} that the reason why the academic community has
failed so far to prove or disprove most of the fundamental claims for advantages
of DSLs lies in the poor quality of the overall tooling and of the DSLs used
developed by academic research. This notion is shared by us.

Nonetheless, modern DSL workbenches have provided the means for experimenting
with the construction of DSLs in a systematic manner. Based on the work of
Freire \emph{et al.}~\cite{Freire14} on formalizing software engineering
experiments, H\"aser and his colleagues have proposed in~\cite{Haser16} a
framework in MPS that allow defining and conducting experiments for validating
DSLs. Very recently, work has started to emerge on concrete frameworks for the
evaluation of usability during DSL development~\cite{Barisic18} and DSL
maintenance~\cite{ThanhoferPilisch17} -- thus responding to the issues
identified by Mernik~\cite{Mernik17} and that are illustrated in
figure~\ref{fig:DSL_design_phases}.

\paragraph{Summary}
From the condensed state of the art above, we can identify the following tends:

\begin{itemize}
  \item For almost two decades professional DSL workbench builders have
  consistently reported case studies of application of DSLs to industrial
  problems, some of those case studies being highly successful.
  The same DSL builders do nonetheless mention that obtaining the data for
  validating the DSL after shipment is difficult~\cite{Tolvanen018}.
  \item During that same period academia as struggled to provide evidence that
  the promises of DSLs in terms of increased productivity hold. Only recently,
  partly due to the arrival of mature DSL workbenches produced by professional
  software, is actual quantification of properties of DSLs such as e.g.
  usability becoming possible. Experimental validation of DSLs based on
  established theory is still under-explored~\cite{Mernik17}.
  \item The DSL workbenches developed by academia are, thus far, insufficient to
  scientifically validate the premises of the DSL-based software development.
  This is on the one hand due to the poor quality of the tooling
  produced by academia~\cite{TolvanenKelly2016}, and on the other hand to very
  low access to real industrial use cases. Most of the surveys we describe above use as study
  material other academic papers, leading to starvation of information on real
  DSL commercial usage.
  \item With the notable exceptions of the work of Tolvanen and Kelly and
  Völter mentioned above in this section, the bridges between the academia and
  industry in the DSL domain are brittle.
\end{itemize}

\section{Proposed Topic of Research} 
\label{sec:topic}

Following section~\ref{sec:soa}, it is clear to us that a
substantial gap exists between academic research in DSLs and their usage in
practice. We wish to address this gap with our research. We thus
formulate our main research question as:\\

\textbf{How can we identify, measure and use characteristics of DSLs to support
their sustained and sustainable adoption by the industry?}\\

Such a research naturally leads to a number of sub-research questions, namely:

\begin{itemize}
  \item What are the characteristics of a DSL that can be quantitatively
  assessed in order improve the chances of a real increase in productivity by
  the industry?
  \item Which tooling is adequate to do such quantitative assessment?
  \item How can those characteristics be summarized in a way that is convincing
  for software-agnostic staff (domain experts or management), leading to the
  qualification of the DSL development process?
\end{itemize}

Regarding the tooling used to perform the quantitative assessment, we are firmly
convinced that areas such as formal methods and machine learning have an
important role to play. This outlook stems from evidence collected 
by our past work (see section~\ref{sec:context4dev}). We thus formulate a
 corollary research questions as: how can formal methods and machine learning help
in the qualification and production phases of a DSL?

Our mid-to-longterm goal is to inseminate the academic community with
the practical experiences and struggles of making DSLs used practice, leading
to the theoretical establishment of the domain.

\section{Context for Development of the Research}
\label{sec:context4dev}

We have significant experience in the design and implementation of DSLs for
areas such as software testing~\cite{lucio08}, model
transformations~\cite{BarrocaLAFS10}, verification~\cite{OakesTLW15,kanav18},
process management~\cite{LucioARAKH17}, software refinement~\cite{Syriani19} and
embedded sytems~\cite{Lucio18}.
These languages were developed using a variety of DSL workbenches, e.g. the
Eclipse EMF framework, MPS or AToM3. Some of the DSLs were developed for
academic purposes, while other were or are being developed with and for
industrial partners such as Diehl Aerospace, Airbus and Rolls-Royce.
 
The role that fortiss itself plays and will play on this research is not to be
understated: we currently have the kind of access to projects and industrial and
academic partners that occurs very seldom in academia or industry. This access
to real industrial settings, with their time and budget constraints and expert
knowledge, will be determinant in the success of the research we propose here.
Also, we cultivate relationships with our customers, meaning it is likely they
will be ready to partly share their DSL validation data with us.

Not to be understated also is the existing body of research on the topic we are
proposing. Although the gap between academia and industry is clear
as per section~\ref{sec:soa}, exciting results on validating DSLs such as the
ones recently presented by Lara \emph{et al.}~\cite{LaraGuerra13}, Freire \emph{et al.}~\cite{Freire14}, Haser
\emph{et al.}~\cite{Haser16} or Barisic \emph{et al.}~\cite{Barisic:12} are
starting points for our research. Also, systematic work by
Mernik~\cite{MernikHS05, Kosar2012, Mernik17} on assessing the state of the art
as well as the clear-cut reports from the industry from Kelly and
Tolvanen~\cite{Tolvanen016, Kelly2009, KellyTolvanen09, Kelly:2013,
TolvanenKelly2016, Tolvanen018} and V\"olter~\cite{Voelter09} provides an
excellent basis for this research. Finally, our large academic network will
allow us to continue collaborating with other scientists in order to improve and
validate our scientific results.

Regarding the usage of machine learning during this research, we have already
started efforts in this direction through the MAGNET project. This project aims
at using machine learning to improve the usability and learnability of the
AutoFOCUS3 tool~\cite{af3}, a collection of DSLs for the development of embedded
systems. The same can be said for formal methods: beside our background in the
domain, the running CBMD project focuses on using a model checker as the
semantic domain for a contract DSL (used at PROTOS GmbH) for runtime
verification~\cite{kanav18} .
Additionally, in the past year we have produced a survey on the usage of machine
learning in formal verification~\cite{Amrani18} as well as a survey of machine
learning in requirements engineering~\cite{Iqbal18}.

In the annexes of this document we provide detailed information about the
context in which our research will be developed:

\begin{itemize}
  \item Synergies with groups at fortiss (appendix~\ref{app:synergies})
  \item Completed and running projects at fortiss (appendix~\ref{app:projects})
\end{itemize}

\pagebreak
 
\begin{appendices}

\section{Synergies with groups at fortiss}
\label{app:synergies}

Domain-Specific Languages are a cross-cutting theme at fortiss. All
competence fields at fortiss deal with DSLs in one form or the
other, even or despite doing it unknowingly.

\begin{itemize}
\item The obvious synergy of our research topic is with the Model-Based
Software engineering (MbSE) group:
MbSE has a model-centric view on systems and software engineering and builds languages using model-driven
techniques. Its main current theme is design-space exploration and the modelling
of  embedded systems, but other themes such as requirements engineering, formal
methods or security are also being pursued. DSLs can be of great assistance to
MbSE by bringing systematic notions of language construction and evaluation to
the development table. A collaboration that is already ongoing is the
exploration of machine learning (in the context of the MAGNET project) to deliver support to users of AF3.
\item Common work is also ongoing with the Autonomous Systems (AS) group, in the
form of the FaktorBUILD project proposal to be submitted to the BMBF. Some of
the work of AS relies on probabilistic programming to model situations where incomplete knowledge must be reliably used
to make decisions in real-time. Here, we help in constructing DSLs at an
adequate level of abstraction such that the IDE provided to users of factor
graphs can leverage good abstractions, static analyses and formal methods to
increase the productivity of factor graph programmers.
\item The collaboration with the Industry 4.0 (i4) group has been so far very
successful, having lead to the implementation of a tool
providing DSLs to express industrial capabilities. Such capabilities, or
skills, can be subsequently be matched to automatically automatically synthesize
controllers for industrial machines. Upcoming work for 2019 in the area will aim
at connecting the completed DSL-based tool to AutomationML and 4Diac in order
to connect the work with outside formats while providing simulation capabilities. 
\end{itemize}

\section{Completed and Running Projects at fortiss}
\label{app:projects}

\textbf{Completed projects:}

\begin{itemize}
  \item IETS3 (DSLs for requirements engineering)
  \begin{itemize}
    \item Role: project leader
    \item Consortium: fortiss, itemis, ZF, Diehl aerospace, Rolls-Royce
    \item Running time and personnel: 2 years / 3 people
  \end{itemize}
  \item CELT (Verification of model transformations via contracts and formal
  methods, developed in MPS)
  \begin{itemize}
    \item Role: project leader
    \item Consortium: fortiss
    \item Running time and personnel: 6 months / 2 people
  \end{itemize}
\end{itemize}

\textbf{Running  projects:}

\begin{itemize}
  \item CBMD (Contracts expressed as DSLs for runtime verification of control
  software)
  \begin{itemize}
    \item Role: project leader
    \item Consortium: fortiss, PROTOS, itemis, SQMi, University of Augsburg
    \item Running time, personnel: 2 years / 2 people
  \end{itemize} 
  \item MAGIC (DSLs for industrial automation)
  \begin{itemize}
    \item Role: project leader
    \item Consortium: fortiss, University of Montréal
    \item Running time and funding: 2 years / 3 people
  \end{itemize} 
  \item MAGNET (Machine learning for usability and learnability in AutoFOCUS3)
  \begin{itemize}
    \item Consortium: fortiss
    \item Running time, funding and personnel: 6 months / 8 people
  \end{itemize}
  \item ARTEMIS (DSLs for engineering and formal verification of software
  requirements for fighter jets)
  \begin{itemize}
    \item Role: project leader
    \item Consortium: fortiss, Airbus
    \item Running time, funding and personnel: 6 months / 2 people
    (Levi + HiWi)
  \end{itemize} 
  \item BaSys4.0 (industrial automation)
  \begin{itemize}
    \item Role: developer of DSLs to describe and operationally match industrial
    capabilities
    \item Consortium: fortiss, Festo, Kuka, ABB, among others.
  \end{itemize}   
\end{itemize}
 
\end{appendices}
\newpage
\bibliographystyle{plain}
\bibliography{group_proposal}

\end{document}
 